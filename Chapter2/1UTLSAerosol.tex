\section{UTLS Aerosol}

The atmosphere of earth is a complex and complicated system that is not completely understood and effects and is effected by the human activities. In the late $18^{\text{th}}$ a theory had been raised that hypothesized that as that as altitude increased the temperature of the atmosphere must decrease and eventually go to absolute zero. This lead to a series of balloon campaigns in the late $19^{\text{th}}$ to discover this mysterious point in the atmosphere, however during these flights it was found that at approximately 12~km an inversion point was discovered where the temperature started to increase and thus the tropopause, which separates troposphere and the stratosphere, was discovered \citep{Hoinka1997}. The region surrounding the inversion point in the temperature profile is known as the Upper Troposphere and Lower Stratosphere (UTLS) which typically extends from approximately 10~km to 25~km. It is a major region of study due to aerosol and ozone concentrations, stratospheric circulation interaction with the troposphere \citep{Baldwin2007}, as well as aerosol effect on water concentrations and its effects on the formations on clouds \citep{Demott1997}. These effects have a large effect on the chemical and radiative balance of the UTLS and thus the planets climate \citep{McCormick1995,Solomon1999} however, uncertainly on their exact alteration to the global climate balance is not currently well known requiring the need for better aerosol measurements \citep{Solomon2007}.

One type of aerosol found located in the UTLS is sulfate aerosol which is produced in the troposphere though both natural and anthropogenic methods. Natural sources of sulfur include emissions from volcanic eruption and marine phytoplankton \citep{Bates1992} where as anthropogenic sources primarily include the use of fossils fuels and burning of biomasses. These source inject OCS, CS$_{2}$, and SO$_{2}$ are released in the troposphere and oxidize to from aerosol in the form H$_{2}$SO$_{4}$ droplets. Nucleation occurs and the droplets grow to a size on the order of 0.05 to 0.20\,\si{\micro\metre} \citep{Brock1995} which affects how the particle scatters light and thus the radiative balance. These aerosols cause incoming light from the sun light to be scattered away from the planet thus cooling the surface temperatures \citep{Dutton1992,Pollack1976}. Aerosols in the UTLS, apart from scattering incoming radiation directly, can also alter the formation of clouds. Aerosol particles in the UTLS can form cloud nuclei from condensation from water vapour. The increase in cloud nuclei tend to increase cloud cover of the planet which leads to an increases the shortwave albedo of the planet \citep{Charlson1992}. The albedo is the measure of the of the fraction of the incoming shortwave radiative that is reflected back into space. A greenhouse effect can also occur from aerosols scattering outgoing InfraRed (IR) radiation from the earth \citep{Kiehl1993}.

All of these previous effects are sensitive to aerosol number density and as well as particle size distribution, combined known as aerosol microphysics. Particle size is quite variable over time depending on the sources of the sulfate aerosol as well the the condensation and evaporation effect it undergoes during its lifetime \citep{Bingen2004}. A single mode lognormal particle size distribution has been used to fit the observed measurements during low aerosol loading periods, that is with little volcanic activity, and is given in the form
\begin{equation}
\label{eq:2.1:AerosolParticleSizeDistrabution}
\frac{dn(r)}{dr}=\frac{n_{0}}{r\ln\sigma_{g}(2\pi)^{1/2}}\exp\left(\frac{-\ln^{2}(r/r_{g})}{2\ln^{2}(\sigma_{g})}\right)
\end{equation}
where $n_{0}$ is the aerosol number density, $r_{g}$ is the mode radius, and $\sigma_{g}$ is the mode width \citep{Jager1991}. Some consideration has been given to using a bimodal distribution after strong volcanic activities where the aerosols ejected from the eruptions have a different particle size distribution that the background aerosol \citep{Deshler2003} but will not be focused on in this work. All three of the parameters of aerosol are necessary in determining its climatological effect and how they chemically react with other species in the atmosphere.

TODO: Add a paragraph about climate effects
