\section{Methods of Measuring Aerosol}

Two fundamental methods are used to measure aerosols concentrations within the atmosphere. The first of these methods are ground based and in-situ measurements which give good detail and information about a specific localized. However, these measurements are limited in scope has they do not have global coverage that is inherent in satellite instrumentation. Both ground base and satellites have important roles in monitoring the planets aerosol content, however each of these methods have inherent advantages and disadvantages. A overview will be given of some of the common methods to determine aerosol concentration and why using different methods helps to increase the overall accuracy and precision of data sets.

\subsection{In-Situ}

In-situ measurement have occurs on balloon based platform in which particle counters are used. Balloon instruments that use particle counters during the assent direct count the aerosol particle as well can determine the particle size distributions. One such instruments is the Optical Particle Counter (OPC) is an active light source instrument that has a incandescent light source internal to the device. The instrument has been launched from Laramie, Wyoming since 1971. The instrument measures the internal light source using the forward scatter at a central angle of 25\si{\degree} over a approximately 30\si{\degree} solid angle to determine aerosol extinction and particle size \citep{Rosen1964, Deshler2003}. A second type of in-situ balloon instruments uses a passive light source, including the sun, moon, or stars, to determine aerosol extinctions. Instruments that use this type of technology are the Absorption par les Minoritaires Ozone et NO$_{\text{x}}$ (AMON) from 1992 to 2003 and Spectroscopie d\si{\arcminute}Absorption Lunaire pour l\si{\arcminute}Observation des Minoritaires Ozone et NO$_{\text{x}}$ - Nacelle 2 (SALOMON-N2) from 2007 onwards which use starlight and moon light respectively \citep{Berthet2002}. In-situ measurements of aerosol extinction give direct measurement of scattered light from the altitude that the balloon is currently situated, which helps to reduce scattered light contamination from aerosol particles at different altitudes especially for instruments that use an active light source. This allows for a most possible isolated aerosol measurements  and direct measurement of aerosol extinction and cross section unlike remote sensing applications from satellites. However, these types of instrument due not active global coverage of aerosol measurements and only give aerosol extinction from a very localized region, like OPC, or have very few flights, for example AMON which had a total of six stratospheric balloon flights, three mid-latitude northern and three high-latitude northern flights. In order to achieve full global coverage satellite remote sensing instruments were create to fill the spatial gap.

\subsection{Occultation}

The method of measuring the atmosphere through occultation, which is looking at the sun though the atmosphere as can be seen in \autoref{fig:2.2:OccultationGeometry}, had been used for a longest in space borne satellite missions and is considered the most tested and robust method of remote sensing. It acquires a vertical profile of the atmosphere by looking at the sun either during the sunrise or sunset. This allows each individual scan to have a high SNR as well be a self calibration for each profile since the instrument can directly look at the sun above the atmosphere to direct measure the sun\si{\arcminute}s intensity and determine a reference spectrum either at the end or beginning of each scan. Since the instrument simply measures the attenuation of the light from the sun cause by the atmosphere, or the optical depth, a simple logarithmic ratio of the measured spectrum with respect to the reference is all that is need to perform the aerosol retrieval. Furthermore the retrievals are coupled together though what is known as the onion-peel method, where the aerosol extinction from the higher altitudes are used to determine the extinction at lower altitudes, thus determining an accurate vertical aerosol profile. The major drawback to occultation satellites is the number of measurements it can record in a single day since the instrument needs to be viewing a sunrise or sunset event which limits the geometry to only 16-48 measurements per days depending on the orbit.

\begin{figure}
    \includegraphics[width=1.0\textwidth]{./Images/2-2-OccultationGeometry.pdf}
    \caption[Occultation Geometry]{The occultation instrument monitoring the atmosphere by scanning the atmosphere by looking directly at the sun.}
    \label{fig:2.2:OccultationGeometry}
\end{figure}

Over the past few decades many occultation instruments have been placed on satellites with a few being Stratospheric Aerosol Measurement (SAM) II from 1978 to 1993, Stratospheric Aerosol and Gas Experiments (SAGE) I \citep{McCormick1979} from 1979-1981, SAGE II \citep{McCormick1987} from 1984 to 2005 , and SAGE III \citep{Thomason2003} from 2001 to 2006. These instruments gave vertical aerosol extinction profiles at multiples wavelengths (except SAM II which only has one wavelength) on a global scale over a long period of time making the data very valuable for validation studies, and trends analysis.

\subsection{LIDAR}

Through the transmission of a laser pulse though the atmosphere, a method known as LIght Dectection and Ranging (LIDAR) can determine aerosol concentrations through the measuring of the intensity of the backscattered laser light at different wavelengths and polarizations. Initially LIDAR was primarily used at ground based facilities to measure aerosol layers dating back to the 1960s \citep{Fiocco1964}. More recently LIDAR instruments have been used on satellite missions including the Ice, Cloud, and land Elevation Satellite (ICESat) from 2002 to 2010 \citep{Schutz2005} with a second mission, ICESat-2, planed for launch in 2017 and Cloud-Aerosol Lidar and Infrared Pathfinder Satellite Observations (CALIPSO) which launched in 2006 \citep{Winker2007}. Traditional LIDAR instruments have looked in the nadir direction (straight down or up) however some instruments have looked slightly off-nadir, as seen in \autoref{fig:2.2:LidarGeometry}, to increases to signal to noise.

\begin{figure}
    \includegraphics[width=1.0\textwidth]{./Images/2-2-LidarGeometry.pdf}
    \caption[LIDAR Geometry]{LIDAR instrument showing a measurement in both the nadir and offnadir lines of sight.}
    \label{fig:2.2:LidarGeometry}
\end{figure}

CALIPSO is a joint mission developed between the National Aeronautics and Space Administration (NASA) and the Centre National d'Etudes Spatiales (CNES) of the United States and France respectively. It uses a two wavelength polarized LIDAR system to achieve high resolution aerosol and cloud retrievals along the satellite's orbital track, it views the ground track initially at 0.3\si{\degree} off-nadir to remove specular reflection from still water and then in 2007 the angle was increased to 3\si{\degree} to reduce effects from orientated ice crystals \citep{Hu2009}. CALIPSO data is used to determine aerosol extinction profiles at 40~km horizontal resolution and 120~m vertices resolution with global coverage from 82\si{\degree}S to 82\si{\degree}N \citep{Young2009}. However, despite the excellent vertical resolution given by LIDAR satellite instruments they suffer from poor Signal to Noise Ratio (SNR) especially during the during daylight hours and for high altitude aerosol measurements \citep{Kacenelenbogen2011}.

\subsection{Limb Emission}

The limb emission geometry the instrument views the atmosphere from looking though a vertical profile as seen in \autoref{fig:2.2:LimbEmissionGeometry} and records radiation emitted from particle in the atmosphere in the microwave range in the instrument line of sight to be able to determine concentration of various atmospheric constituents which allows measurement to be taking during night and day to achieve global coverage. Satellites that us the limb emission technology include Michelson Interferometer for Passive Atmospheric Sounding (MIPAS) \citep{Fischer2008} from 2002 to 2012 and Microwave Limb Sounder (MLS) \citep{Waters2006} from 2004 to present. These instruments gives vertical resolution of 2-5~km for volcanic sulfate \citep{Thomas2010} and are unable to detect background sulfate aerosols. The volcanic sulfate that are not removed by atmospheric transportation will oxidize into sulfate aerosols. A major drawback to the limb emission method is that precise information on pointing needs to be known to be able accurately determine concentrations as well since the emission lines from particle depends on the temperature an accurate temperature and pressure profile of the atmosphere must also be known in order to determine accurate results \citep{Von2003}. The geometry of the problem combined with the the accurate knowledge needed for the pointing and temperature and pressure profiles requires the used of a forward model to accurately determine the emission lines from species at different pressures and temperatures to be able to determine the volcanic aerosol from measurements \citep{Livesey2006}

\begin{figure}
    \includegraphics[width=1.0\textwidth]{./Images/2-2-LimbEmissionGeometry.pdf}
    \caption[Limb Emission Geometry]{Geometry of an instrument preforming limb emission spectroscopy. The solid black line is a photon emitted from a volcanic sulfate particle.}
    \label{fig:2.2:LimbEmissionGeometry}
\end{figure}

\subsection{Limb Scatter}

The limb scatter technique uses the same geometry as limb emission viewing the atmosphere from the side and above the surface of the earth rather than looking directly at the sun and views scattered light from the sun that enters the instruments line of sight, however measurements can only be taken in the sunlit atmosphere. The light can be scattered into the line sight through single scatter or multiple scatter. Single scatter is where light from the sun is interacts with a particle in the atmosphere and scatters it directly into the line of sight. Multiple scatter is when the photon of light undergoes sever scattering events before entering the line of sight including scattering off of multiple particle in the atmosphere before entering the line of sight or scattering off of the ground into a scattering event into the line of sight. These event can occur any number of time before entering the systems line of sight. The tangent point on the limb is the point where the distance between the line of sight and the surface of the earth is minimized. Although this method of measurements give good gives good vertical resolution it is an inherently complicated measurement due to the complex nature of scattered light. Multiple scattered light consists of 10-50~\% of the signal depending of the specific geometry  and modeling the signal which requires considering all points in the atmosphere that the multiple scatter events can occur instead of just considering the atmosphere in the line of sight \citep{Oikarinen1999}. Limb scattered measurements can contain a lot of valuable information about species that scatter in the atmosphere but due to the nature of the complexity of the measurement, the retrieved values relies on the ability to successfully and accurately model the measurement. Currently two types of limb scatter instrument are in operation around earth, scanning and imaging, and a brief description of each will be covering the the following sections.

\begin{figure}
    \includegraphics[width=1.0\textwidth]{./Images/2-2-LimbScatterGeometry.pdf}
    \caption[Limb Scatter Geometry]{Limb scattering geometry measurement for an instrument where single and multiple scattering events occur.}
    \label{fig:2.2:LimbScatterGeometry}
\end{figure}

\subsubsection{Scanning}

The first type of limb scatter instrument are known as scanning instrument since these types of instruments can only look at one location at a time and must move the line of sight vertically in order to get a vehicle profile. This movement is done by one of two methods: using a moving mirror or nodding of the satellite to alter the line of sight. These instruments get good vertical resolution on the order of 1-2~km with along the track horizontal resolution approximately 500~km due to the scanning nature and the amount of time it takes to gather a whole profile. Two instruments that use the scanning technique at the SCanning Imaging Absorption spectroMeter for Atmospheric CHartographY (SCIAMACHY) \citep{Bovensmann1999} Optical Spectrograph and Infra-Red Imaging System (OSIRIS) \citep{Llewellyn2004}.

The OSIRIS instrument records measurements between 82\si{\degree}N and 82\si{\degree}S at 0600 and 1800 local time with tangent altitudes from 5 to 100~km. Measurements are retrieved into 750~nm aerosol extinctions using the SASKTRAN radiative transfer foreword model (discussed in \autoref{sec:4.1:Sasktran}) with a nonlinear inversion technique \citep{Bourassa2007}. The current 750~nm aerosol product has near global coverage and an uncertainty better than 25\% from 20 to 30~km \citep{Bourassa2012a,Bourassa2012b}.

\subsubsection{Imaging}

Ozone Mapping and Profiler (OMPS) \citep{Dittman2002} and Atmospheric Limb Tracker for the Investigation of the Upcoming Stratosphere (ALTIUS) \citep{Dekemper2014}. 