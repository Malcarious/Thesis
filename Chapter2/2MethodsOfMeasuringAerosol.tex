\section{Methods of Measuring Aerosol}

Two fundamental methods are used to measure aerosols concentrations within the atmosphere. The first of these methods are ground based and in-situ measurements which give good detail and information about a specific localized. However, these measurements are limited in scope has they do not have global coverage that is inherent in satellite instrumentation. Both ground base and satellites have important roles in monitoring the planets aerosol content, however each of these methods have inherent advantages and disadvantages. A overview will be given of some of the common methods to determine aerosol concentration and why using different methods helps to increase the overall accuracy and precision of data sets.

\subsection{In-Situ}

Deshlers Balloons and SOLAMON

\subsection{Nadir}

make sure you discuss CALIPSO and discuss resolution capabilities of the stratospheric product

\subsection{Occultation}

SAGE II SAGE III SAM II

\subsection{Limb Emission}

MIPAS and MLS (I think) now have sulphate products

\subsection{Limb Scatter}
\subsubsection{Scanning}

SCIAMACHY OSIRIS

\subsubsection{Imaging}

OMPS and ALTIUS 