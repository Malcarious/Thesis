\section{BaffleDesign}

A BASIC START:

Many questions arose when the baffle for the ALI system was devised. Questions arose over length and width of the baffle as well as how many baffles should located in the final design as well as the interior shape of the baffle layout. These were all important questions that needed ansering since stray light rejection from the ALI insturment is crutial in order to achieve high sensitivity of the light from the atmoshere.

The baffel system is designed such that all light entering the system hit at least 3 surfaces before it can enter the opical system for direct scattering and 2 surfaces for indirect scattering. This methoid is called the thriple bounce disign and i standeredly used in opticals to minimise stray light. In the system the baffels are spaced in such that no stray light that can enter the system is minimized by having the stray light hit three incoming surfaces reducing the overall intersity of the light.

The first point of the discussion was a hight versus width discussion. IN a baffle system the larger the baffle is by shear volume the better the baffle can be designed to reduce stray light. However there is a limited amount of space to build the ALI instrument and the baffle must share space with optics, eletronics and power systems; as such a size had to be selected. A height and width of 70 mm was choosen. The CCD camera used in the design had a height of a little greater then 70 mm and did not want the instrument to have to be any taller. This left the length of the baffle to be determined. The length is also limited by the space for ALI as well as the field of view and entrance apperture size, and its location. One always needs to make sure the the optical stop is the location that limits the amount of light that is entering the system. If this is artifically changed with the baffle design it will affect the preformance of the instrument itself. If the optical stop is moves further from the optical system one keeps the same field of view but limits the amount of light that enters the system. Thus changing the overall F/\# of the system and either increases the exposure times or decreases the signal to noise ratio. The other case is if the optical stop is moved closer to the optical system, which causes the opposite problem which is more light enters the system than the system was designed for causing an excess of stray light and rendering the baffel useless.

This information leaves us with three kinds of fundimental locations to put the optical stop of the system in the design phase. The first is to put the optical stop as close ot the front lens as possible, second is to place the optical stop at the far end of the baffel, and third to place the optical stop in the middle of the baffle system. IN the first case the baffels have a  diverging shape and there are few baffles. In the second case

Although the overall choice between these three do not have  a great effect on the effencey of the baffle itself it has an effect on the shape on the baffle system as well as the number of baffles required. The greatest difference is the change in the optical system required, the further away the optical stop is from the front lens the larger the optical components need to be in order to accept all of the light, which makes the system heavier, larger and more expensive to build. In ALI the optical the stop was choosen to be close to the first lens to make the system has small and compact as possible as well has to reduce the number of baffels that needed to be built for the system and thus the overall cost with reducing the systems effectivness.
