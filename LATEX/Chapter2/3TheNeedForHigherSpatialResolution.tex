\section{The Need for Higher Spatial Resolution}

FROM COMP:

Aerosol number density in the atmosphere is highly variable and difficult to measure due to the unknowns microphysics, which includes particle size distribution and composition. Current satellite capabilities, such as the measurements made by OSIRIS, are limited to approximately 2\.km vertical resolution and sampling every 500\,km along the track.  Much higher spatial resolution is required for current weather and climate model of UTLS processes: desired resolutions are in the range of 200\,m vertically and tens of kilometres horizontally.  Advances in both detector and filtering technologies allow for new instruments to be designed and built with higher resolutions than in previous generations and allow for the capture of two dimensional spacial images. Current limb scatter measurements, like OSIRIS, are preformed using a diffraction grating or a prism which images a spectrum with a single line of sight which is scanned vertically to produce a final product with two dimensions, a spectral and a spacial dimension. However, a filtering device with good imaging characteristics would allow for images to be recorded with two spacial dimensions. The spectral dimension can occur from taking several images in rapid succession at different wavelengths. These measurements will allow for the determination of fine structures in the aerosol extinction profiles and cloud distributions both in the horizontal and vertical direction with required resolution of 4\,km and 0.5\,km respectively \citep{Adams2009}.

Another issue with current instrumentation is limited spectral range. This renders it difficult to determine aerosol particle size. Work done by \cite{Rieger2012} has shown that measurement vectors for limb scatter are only sensitive to different particle size distributions with samples at and beyond 1000\,nm. In order to retrieve any particle size information, measurements will be needed beyond 1000\,nm, where there is some sensitivity to particle size; however, in order to be able to have good sensitivity to aerosol microphysics spectral images recorded at 1200\,nm and beyond are required.

Asian monsoon, resolving high extinction layers, detecting eruption plumes for aviation, etc 