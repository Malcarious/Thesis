\chapter{Future Work and Conclusions}

From Paper:

A description of the prototype ALI using an AOTF for active filtering in the visible to near IR with a telescopic optical layout with the purpose to measure aerosol extinction from the upper troposphere and lower stratosphere with a high vertical and horizontal resolution with images was presented here. The AOTF has fast stabilization times and the ability to disable the filter gives an excellent method to remove stray light from the final measurements. ALI is able to measure aerosol microphysics in remote atmospheric sensing.

ALI was tested on board the CARMEN gondola from the balloon launch facility at Timmins, Ontario. Aerosol extinction profiles were determined and had good comparisons to ORISIS and SALOMON in profile shape. The absolute extinction values are different by a large amounts but could be attributed to the large amount of unaccounted systematics in the retrieval algorithm. Overall ALI preformed admirably and verified the use of this technology for future atmospheric remote sensing missions.

Future upgrades to ALI would include the realignment of the optics for flight temperatures to allow for higher resolution measurements and retrievals. Secondly, replacing the CCD currently used on ALI with a camera with faster readout would allow for a higher quantity of data to be taken by reducing the approximate 25~s readout time down to a smaller value necessary for a satellite missions on the order of 1~s per wavelength. 